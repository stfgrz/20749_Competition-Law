\setchapterabstract{This chapter introduces competition law, focusing on the US and EU frameworks. It explores the economic rationale behind competition law (promoting efficient markets) and the different goals that antitrust aims to achieve (consumer protection, preventing monopolies, etc.). The chapter also discusses historical aspects of competition law, including its origins in the US and the Sherman Antitrust Act of 1890.}
\chapter{Introduction and History}
\vspace{-1.5cm}

{\chaptoc\noindent\begin{minipage}[inner sep=0,outer sep=0]{0.9\linewidth}\section{Antitrust law (and the economic rationale behind it)}\end{minipage}}

    What is antitrust and what is the purpose of antitrust/competition law? 
    At their core, antitrust laws and their enforcement are intended to ensure lively competition in the marketplace to promote the well-being of the economy as a whole.
    Justice Marshall in U.S. v. Topco Association (1972) said:
    \begin{quote}
        «Antitrust laws (…) are the Magna Carta of free enterprise. They are as important to the preservation of economic freedom and our free-enterprise system as the Bill of Rights is to the protection of our fundamental personal freedom». 
    \end{quote}

    \Remark{
    Antitrust law is a set of legal rules aimed at preventing some firms’ practices that may worsen the market well-functioning (as it results from CS negative variations).
    }
    The economic rational (from neoclassical standard economics) behind competition law is the following: 
    \begin{itemize}
        \item Healthy competition amongst rival firms stimulate production and allocative efficiency and thus more products at lower prices (and more quality, variety and innovation, too). 
        \item Restrictive agreements – cartels – between private parties allow firms to control quantity and/or prices to obtain a monopolistic equilibrium, reducing welfare. In other words, cartels lead to a misappropriation of surplus/welfare from consumers to firms (redistributive effect), and to a decrease in total welfare (allocative effect).
        \item This is also true for (2) abuses of a dominant position and (3) (some) mergers.
        \item[\(\Rightarrow\)] Thus, the goal of antitrust laws is to preserve markets’ well-functioning by prohibiting those anticompetitive-practices that worsen TS (and CS in particular), i.e., prohibiting restrictive arrangements such cartels, abuses of a dominant position and (some) mergers. 

    \end{itemize}

    At least, this is the current mainstream interpretation of the US and European antitrust legislation and its goals. An interpretation that has been, and is still, strongly objected by many scholars. 
    \begin{itemize}
        \item Some contend that antitrust is an unfunded religion and that it should be repealed or at least strongly limited, as other forms of oppressive and costly State regulation that constrains the freedom of citizens and firms to act. 
        \item Others (from the «leftist» Brandeis antitrust movement to the right-wing advocates of a more populistic enforcement) suggest that antitrust goals should be widened to protect: 
        \begin{enumerate}
            \item citizens and their rights against big-tech economic giants (like Facebook, Google, Apple, Microsoft, etc.), 
            \item small business against the digital platforms (like Amazon  or Alibaba), but also big supermarket chains and department stores (walmart, target, etc.)
            \item national «strategic assets» and industrial champions from takeovers and or “unlawyal competition” by “foreigners”. 
            \item Internal markets/firms from external (often label “unfair”, “disloyal”) competition.
        \end{enumerate}
    \end{itemize}

    \Example{
    6 February 2019, EC blocked Siemens/Alstom’s proposed transaction.
    \begin{itemize}
        \item Proposed deal would significantly harm competition, reduce innovation and lead to foreclosure of smaller competitors
    \end{itemize}
    The merger would have created the \textit{“undisputed market leader”} in some signaling markets and a \textit{“dominant player”} in the EU market for high-speed trains 
    \begin{quote}
        “The merged company would have become, by far, the largest player in Europe and in some signaling markets there would be no competition left. […] We found that competition from other suppliers would have been insufficient to replace the considerable loss of competition due to the merger. Customers, including train operators and rail infrastructure managers, would have been deprived of a choice of suppliers and products.”   (M. Vestager)
    \end{quote} 
    EC remained sceptical about parties’ argument of a future Chinese competitive threat. The parties were not willing to propose a clear-cut remedy
    \begin{quote}
        "[They] could have obtained our approval for the merger if they had proposed appropriate remedies to address our competition concerns” (M. Vestager)
    \end{quote}
    }{Siemens / Alstom}

\newpage
    \subsection{The French-German Manifesto}

        After the EU Commission blocked the merger between Siemens and Alstom, France and Germany published a manifesto (19 February 2018) calling for the EU merger rules to be changed.  “Today, amongst the top 40 biggest companies in the world, only five are European,” the manifesto reads.  The French and German Ministers of the Economy underlined what they termed a “massive disadvantage” of European companies in comparison to their Chinese and American counterparts. 
        \textit{“In order to be able to create a new European champion, it is necessary to change the European regulatory framework”}. The 2 European countries suggested updating merger guidelines to take greater account of competition at the global rather than the European level. 
        The French Minister Le Maire said: \textit{“We have powerful, modern technologies and we don’t want them to serve other continents than ours.” “Sometimes there is nothing more stupid than European rules” }he said. 
        \textit{“It’s all about keeping the created value in Europe,”} Le Maire said.  \textit{“You don’t want to see a company that has been funded for many years gone into the hand of investors,”} he added, referring to the 2016 purchase of German robotics maker Kuka by China’s Midea, a move that created much controversy in Germany. 

        \Note{
        \textcolor{royalpurple}{\textbf{A Panacea per tutti i mali | }}
        Antitrust is sometimes considered a miracle cure. It should fight unemployment, inequality, discrimination, breach of fundamental rights, democracy, pollution. These problems are mainly linked to bigness and excessive market power of GAMAF, and many other industry giants (for example vis-à-vis their workers).   
        }
        
        Thus, antitrust should be applied to limit market power and favour these different goals. However, antitrust failed its mission. 
        \textbf{Why?}\sn{\Note{Is it a question of goals, tools, enforcement? Or is it the wrong tool to accomplish that goal? Or is it that the industry giants are so strong that were able to limit antitrust enforcement to maintain their power?}}
        The discussion is open. The answer is not a “black or white” one. 
        However, some truths: 
        \begin{itemize}
            \item Enforcement has been quite lenient in the past 20 yrs;
            \item Focus on static effects on prices leads to under-enforcement;
            \item It is very difficult to obtain many different goals with the same tool; 
            \item Antitrust is an ex post system that is not always fit for rapidly evolving market.\sn{\Remark{\textbf{Political answer}: ex ante regulation in some specific sectors, like main internet digital platforms.}}
        \end{itemize}
        

        \Example{
        EC imposed a total €8.25 billion in fines on Google
        \begin{itemize}
            \item \textbf{Google Shopping}: EC fined Google €2.42 billion for giving an illegal advantage to its own comparison-shopping service, 27 June 2017.
            \item \textbf{Google Android}: EC fined Google €4.34 billion for imposing illegal restrictions on Android device manufacturers and mobile network operators, 18 July 2018.
            \item \textbf{Google AdSense}: EC fined Google €1.49 billion for engaging in illegal practices in online advertising, 20 March 2019.
        \end{itemize}
            EU Commission is still “looking” into other Google’s anti-competitive behaviour, but the answer arrived too late and the proceedings were too slow to maintain competitive pressure in the market. 
        }

    \subsection{Why do we focus on (US and) EU antitrust law?}
    
        Antitrust law was born in the U.S. in the 1890. The main principles and notions of antitrust law come from the U.S. experience and affected the antitrust experiences of many other jurisdictions around the world. Still nowadays, the legal decisions of U.S. antitrust institutions are landmarks.
    
        Yet, in the last 30 years, the jurisdictions of many countries in South America, Africa and Asia have been reproducing the EU competition law approach, mainly because it is easier to copy and enforce than the U.S. one.
    
        
        \begin{table}[h]
            \centering
            \begin{tabular}{|p{1cm}|p{2.5cm}|p{2.5cm}|p{2.5cm}|}
                \hline
                 & \textbf{The rules to be applied} & \textbf{Those bringing a case are} & \textbf{Those deciding the cases are mainly}  \\
                \hline
                \textbf{US} & Must be inferred from the case law & Private citizens (95\%) and administrative authorities (5\%) & Judges \\
                \hline
                \textbf{EU} & Are mainly stated in pieces of written law & Authorities (90\%) and private citizens (10\%) (changing) & Authorities and Judges \\
                \hline
            \end{tabular}
            \label{tab:my_label}
        \end{table}

        Since the U.S. and EU markets are rich, advanced and innovative, the U.S. and EU antitrust institutions:
            \begin{itemize}
                \item focus on very important cases
                \item deal in advance with many competitive problems that other jurisdictions will face only afterwards (hi-tech, patents, Iprs, etc)
            \end{itemize}
    
            \subsubsection{Is Antitrust a U.S. invention?}
                When you read about antitrust, you sometimes find 2 blackletter statements: 
                \begin{itemize}
                    \item[a.] Antitrust as a “piece of legislation” is a U.S. invention, with the enactment of the Sherman Act in 1890, and 
                    \item[b.] Its main provisions are intended to protect the process of competition and rivalry between firms in order to keep markets efficient and innovative
                \end{itemize}
    
            \Remark{
            \textbf{Is this true? }
            \textbf{No}: While (a) it is true that economics (still) play an important (crucial?) role in antitrust, and (b) the US Sherman Antitrust Act is the first modern set of rules dealing with the competitive behaviour of firms 
            }

\newpage
\section{A very short introduction, for a very long history}

        First, 6 millennia of history of antitrust. To cut a long story short (skipping ancient Sumerian, ancient Greek rules).
        \begin{itemize}
            \item Lex Julia de Annona, enacted during the Roman Empire around 18 B.C. (Augustus). To protect the corn trade, and roman citizens, heavy fines were imposed on anyone “directly, deliberately and insidiously stopping” (and sometimes more cruelly sinking) ships transporting corn from the colonies. 
            \item Ergo: a provision dealing with exploitative behaviour, a behaviour that was heavily sanctioned (political reason behind: merchants becoming richer and richer were a threat to establishment).  
            \item With the crisis of the Roman empire fines and imprisonment were deemed insufficient to stop this immoral exploitative behaviour. As the roman lawgiver already understood, monopolies grow easily but die hard. Under Diocletian, in 301 A.C., an “Edict” on maximum prices established a death penalty for anyone violating a tariff system, for example by stockpiling, concealing or contriving the scarcity of everyday goods. 
        \end{itemize}

            \subsubsection{Christian philosophical thought during the middle ages | Thomas Aquinas}
                Aquinas: it is immoral to raise prices only because a particular buyer has an urgent need for what is being sold and can be persuaded to pay a higher price due to local conditions:
                \begin{quote}
                    “If someone would be greatly helped by something belonging to someone else, and the seller not similarly harmed by losing it, the seller must not sell for a higher price: because the usefulness that goes to the buyer comes not from the seller, but from the buyer's needy condition: no one ought to sell something that doesn't belong to him,” Summa Theologiae, 2-3, q. 77 
                \end{quote}
                \Remark{
                Aquinas would therefore condemn practices such as price discrimination and excessive pricing (many edicts during the Carolingian era). 
                }

            \subsubsection{Many goals (or many souls, not all candid) | Sherman Anti-trust Act 1890 }   
                An Act against the formation of trusts?
                1830-50 Industrial Revolution + Expansion towards West: 
                \begin{itemize}
                    \item[a.] formation of a large single market due to transportation \& communication allowing to expand towards the center and then the western part of the U.S.A.; 
                    \item[b.] economies of scale and scope; 
                    \item[c.] industrial-technical innovations
                \end{itemize} 
                \(\Rightarrow\) expansion in the size of firms especially in some markets (oil, steel, transport, freight, manufacture); 
                Fall in transportation and communication costs resulted also in a rise of (geographic) competition \(\Rightarrow\) lower prices.

            \begin{itemize}
                \item Due to large investments, firms’ attempt to operate at full capacity to cover fixed costs resulted in pressure to sell and price decreases.
                \item At the time, heavy crises due to drought, etc., heavily affected agriculture and livestock (i.e., the most important economic sectors at the time, employing most of the US workforce), small businesses in fast growing cities, and thus the demand for goods and services (like oil, transport, etc.).\sn{\Remark{\textbf{The result}: Supply shifts to the right \( \to \) Demand fluctuate }} 
            \end{itemize}

            

        \subsubsection{Price wars and agreements to defend prices}
            Price wars became the rule. Bigger firms reacted to price wars with diverse strategies: 
            \begin{itemize}
                \item \textbf{Collusion}, i.e., price agreements to keep margins high. But... as Rockefeller, the owner of the Standard Oil Co., said: agreements should not be like “ropes of sand”, that the first wind and the morning sun dissolve.  
                \begin{itemize}
                    \item If firms agree to fix prices but they cannot monitor each others’ conduct, in case of shocks (e.g., if demand weakens), all firms are pushed into cheating and the agreement breaks.
                    \item However, with the formation of “trusts”, things would be different. Agreements would be much more stable (reciprocal monitoring and cross-shareholding). Why?
                \end{itemize}
                \item \textbf{Predatory pricing}, i.e. a strategy adopted by the bigger companies was to eliminate competition by acquiring or excluding competitors from the market). 
            \end{itemize}

            \Note{
            There is a third strategy: \textbf{pushing competitors out of the market via market power}. E.g., after a couple of months of predatory pricing, big companies will still have the liquidity required to survive and carry on business. Instead, smaller companies will run out of liquidity quicker and go bankrupt.
            }

            As a result of these strategies, consumers and small businesses were hurt by high prices. 
            A large number of SME were pushed out of the market. Farmers and small businesses were economically weak, but politically strong because very copious as voters (backing by the so-called agrarian movement).
            Fear that the economic power of bigger firms could become unstoppable, and conditioning over political power by way of influencing laws and measures in favour of the bigger companies, trusts, and banks.  
            In the 1880s, there was enough consensus for a Federal Act against trusts. In 1888 all the political parties were sponsoring an anti-trust law in their general election agenda, and in 1890 the Sherman Anti-trust Act was adopted.

    \subsection{Sherman Antitrust Act 1890}

        \subsubsection{Section 1}
            \begin{quote}
                “Every contract, combination in the form of trust or otherwise, or conspiracy, in restraint of trade or commerce among the several States, or with foreign nations, is declared to be illegal. Every person who shall make any contract or engage in any combination or conspiracy hereby declared to be illegal shall be deemed guilty of a felony, and, on conviction thereof, shall be punished by fine not exceeding \$10,000,000 if a corporation, or, if any other person, \$350,000, or by imprisonment not exceeding three years, or by both said punishments, in the discretion of the court”.
            \end{quote}

        \subsubsection{Section 2}
            \begin{quote}
                "Every person who shall monopolize, or attempt to monopolize, or combine or conspire with any other person or persons, to monopolize any part of the trade or commerce among the several States, or with foreign nations, shall be deemed guilty of a felony, and, on conviction thereof, shall be punished by fine not exceeding \$10,000,000 if a corporation, or, if any other person, \$350,000, or by imprisonment not exceeding three years, or by both said punishments, in the discretion of the court"
            \end{quote}    

            \Example{
            \textcolor{dgreen}{\textbf{Standard Oil Co. v. US 221 U.S. 1 (1911)}}. Company sanctioned and divided in many regional branches (among which The 7 sisters !).
            }

        \subsubsection{The Sherman Act and its original goals}
            At the time, in the discussion of the bill there were some references to the need to promote competition and thus lower prices, but in the wording of the act you could not find any specific reference to efficiency, welfare maximization, or consumer surplus. 
            The reasons behind the enactment of the Sherman Act were manifold:  
            \begin{itemize}
                \item Control the (excessive) economic power by big companies (political antitrust, something we have heard of again very recently, see Google, Facebook, Apple, Amazon, and the likes);
                \item Protect small businesses and entrepreneurs’ “right” or “freedom” to enter (and stay) in the market (freedom to act, fairness, etc.) 
                \item Protect categories that also represented most voters at that time (consumer protection/“populism”).  
            \end{itemize}

            Things gradually changed over time and the interpretation of the law changed too, allowing for a more economically oriented interpretation of antitrust goals and provisions. Antitrust law provisions and notions are very general and open-ended. Think, for example, to article 101 concepts like “undertaking” “restriction of competition”.
            Many factors can influence their interpretation and enforcement:
            \begin{itemize}
                \item \textbf{Economic cycles} and downturns;
                \item \textbf{Economic thinking} and its evolution over time
                \item \textbf{Politics} (especially when dealing with mergers and dominance)
            \end{itemize}

            \Note{
            \textcolor{royalpurple}{\textbf{Competition goals and enforcement may vary over time according to the economic/political situation of a country |}}
            U.S. Supreme Court describes the Sherman Act in a 1958 opinion as a \textit{“comprehensive charter of economic liberty aimed at preserving free and unfettered competition as the rule of trade.”} Northern Pacific Railway Company v. United States, 356 U.S. 1, 4 (1958).
            }
            
            But during the great depression: less antitrust enforcement, more price control, more regulation (new deal). The answer at the time was \textbf{crisis cartels.}\sn{\Definition{In the economics literature, the term ‘crisis cartel’ can be used in reference to two very different phenomena. The first is where a government allows or fosters coordination or agreement between firms during an economic downturn. The second is where a cartel emerges during a period of economic downturn without government permission or legal grounds.
            }{Crisis Cartel}}

            \Example{
            \textcolor{dgreen}{\textbf{1933  - Coal Mine industry.}} 
            To avoid possible bankruptcy, 133 Coal Mine producers formed a JV to control prices and allocate output. The Court considered the agreement as a reasonable protection of the market against excessive competition.   
            \begin{itemize}
                \item[a.] The JV … (also) ... encouraged research, advertising, and streamlined distribution;
                \item[b.] The parties faced (some) competition from new fuels and coal in adjacent regions;
                \item[c.] The defendants’ desire to eliminate “destructive practices” was to be considered as a redeeming virtue that could help justify the restraint. The JV merely ameliorated “injurious practices,” which according to the Court “demanded correction.”
            \end{itemize}
            The JV was «an attempt to organize the coal industry and to relieve the deplorable conditions resulting from over expansion, destructive competition, wasteful trade practices, and the inroads of competing industries» Appalachian Coals, 288 U.S. at 366–67.
            }

    \subsection{Economic (and non-economic) Crisis}

        The attitude showed by competition authorities in some recent crisis has been different, at least formally. 
        \begin{itemize}
            \item \textbf{2008 – Credit crunch}. No major changes to competition policy but some procedural adjustment to help saving firms in financial distress. 
            \item \textbf{2020/21 – Covid}. Adjustments to help firms better coordinate, exchange information, and act to solve some market failures, like lack of protecting masks or ventilators at both the manufacturing and distribution stages. 
        \end{itemize}

        \Remark{
        \textbf{EU}: we need resilient supply channels
        }
        
    \subsection{Influence (direct or indirect) of Economic Thinking}

        Economic schools of thought played a fundamental role in shaping antitrust: 
        \begin{itemize}
            \item[a.] \textbf{Harvard school SCP paradigm}: tough enforcement in general, particularly on mergers (Californian supermarkets).
            \item[b.] \textbf{Chicago school (of antitrust)}: very strong influence in general (vertical agreements and the theory of free riding;  attempt to monopolise and monopolisation and the right incentives; vertical and horizontal mergers, etc.; risks of type I errors). If you like markets, let the market correct themselves, enforcement – like regulation - may worsen market structures.
            \item[c.] \textbf{Post Chicago school}: focus on more realistic economic models. Attention on consumer behaviour, etc. 
        \end{itemize}


    \subsection{Antitrust and Politics}

        \subsubsection{AT\&T-Time Warner-CNN Vertical Merger}

            Two main characters:
                \begin{itemize}
                    \item \textbf{Donald Trump} (DT): DT was the Republicans’ candidate to the Presidency of the USA.
                    \item \textbf{Makam Delrahim} (MD): MD was considered a «conservative» Chicago School lawyer, partner in a law firm.  
                \end{itemize}

            \textbf{Context}: A vertical merger between AT\&T and Time Warner. The aim was to offer to AT\&T’s customers Time Warner products, like HBO, WB, Turner, TNT, and CNN. 
            \textbf{Standard economic evaluation}: no direct increase in market power, but possible foreclosure of competitors.

            \textbf{DT}: CNN (part of the Time Warner group) is «fake news»; «the power structure I’m fighting»; «a deal we will not approve in my administration because it’s too much concentration of power in the hands of too few»; «deals like this destroy democracy».
            
            \textbf{MD}: About the ATT Time Warner merger: «I don’t see this as a significant antitrust downside». 
            
        \subsubsection{Developments}

            After gaining the elections, DT appointed MD to the Antitrust Division as A.A.G. (chief);   MD changed its views on the issue and filed a motion to the US District Court – District of Columbia/Washington to enjoin the merger.   Was this a selective enforcement case, i.e. an action brought only after a strong pressure from the White House (DT) to act against ATT to punish the liberal CNN?   As a matter of fact, why this pro-business administration, which claimed to give corporate America lower taxes and looser regulation, tighten the screws on this specific merger?   DT relentless criticism of the network has hung over the AT\&T case from the start, sparking speculation that DT’s personal views and interests, rather than legitimate antitrust enforcement, was fueling the case all alone.   Paradoxically, the Antitrust Division was backed by many “leftist” economists and think-tanks.  

        \subsubsection{Outcomes}

            The AD lost before the District Court, but appealed the decision. The U.S. Court of Appeals in Washington D.C. upheld the lower court’s ruling in favour of AT\&T on February 26, 2019, stating it did not believe the merger with Time Warner would have any negative impact on either consumers or competition. If you are interested in the outcome of the decision, see \href{https://www.cbsnews.com/news/at-t-time-warner-merger-approved-by-u-s-appeals-court-81-billion-takeover-wont-harm-competition/}{here}.

\newpage
\section{The Rise of Competition Legislation in Europe and Japan}

    \subsection{Europe: Same problem, different answer}

        European economic development at the end of the XIX century was not very different from that of the U.S.A, and, even in EU, tensions between economic and political power began to arise.
        The reaction from European States and governments was, however, the opposite of the U.S.A. European States, especially in central Europe, did not aim to break the industrial giants’ growth. Instead, they tried to control and govern them.
        Direct State intervention into the economy and "control" of industrial aggregates were regarded as essential factors for stabilizing economic cycles, avoiding costly price wars, oversupply, and abusive conduct. 

        \subsubsection{In the Wake of WWII}

            The frequent economic crises and wars that shook the European continent in the first part of the XX Century worsened the picture. Cartels, far from being considered illegal, were regulated. 
            In Germany, after the great depression, participation in cartels was made compulsory in many sectors, with representatives of the government sitting on the board of companies to ensure maximum coordination.  
            In other countries, entire industrial sectors went under State control (in Italy, e.g., IRI, ENI).
            The threat of a second WW only increased the degree of State intervention and public control over the economy. Business co-ordination, and the formation of horizontal groups of competing companies were considered the most efficient solution for increasing and converting production in view of the war effort. 

    \subsection{Post WWII | A different scenario in Europe (and Japan)}

        Considering the nasty effects of the “marriage” between industrial giants and governments, economic development and freedom of citizens could be better protected by a system of free competition.
        The Allies imposed the adoption of anti-monopoly rules in countries that lost the WWII. 

        \subsubsection{Germany}

            In Germany, the first "antitrust" legislation (Dekartellierungsgesetze) was issued by General Clay to dismantle the excessive economic power accumulated in the 1930s and considered (in tandem with the totalitarian regime), a formidable apparatus of war and destruction. The first break-ups took place in the chemistry and steel sectors, the flagships of the pre-war German industry. 
            Later on: other reasons to introduce an antitrust legislation by the Americans:  
            \begin{itemize}
                \item A candid one: promote harmonious and democratic economic development in Europe;
                \item A less candid one: reduce the strength of German companies to protect US companies from international competition and facilitate the entry of US companies into the European market.
            \end{itemize}

        \subsubsection{Japan}

            Same action in Japan to limit the strength of family-owned zaibatzu, the multi-market industrial combinations having responsibility in creating the huge Japanese war capacity. In 1947 the US imposed an antitrust law, model upon the US one (with an authority very similar to the FTC).

        \subsubsection{European Countries}

            At the end of the 1950s, many European countries, like France, Germany and England, had produced rules to protect competition. 
            However, those rules pursued quite different objectives: the French law was directed mainly at price control and regulation, 
            The British competition law was more likely to pursue the objective of full employment and protection of British industry and trade, under the public interest label; 
            The German one was founded not only on contrasting excessive concentration, but also on protecting economic freedom and workable markets.

\section{European Treaties}

    \subsection{ECSC | European Coal and Steel Community}

        In this context, Articles 65 and 66 of the ECSC Treaty prohibited agreements which restrict competition in the common market, abuses of dominant positions, and concentrations leading to a dominant position in the market or distorting competition between the Member States.

        Thus, provisions that closely resembled the US antitrust ones were used mainly with peace-keeping purposes. 

    \subsection{TFEU | Treaty on the Functioning of the European Union}

        The same 6 countries, a few years later (1957), reached a wider agreement (EEC or Rome Treaty), that started the European integration process\sn{\Remark{Healthy competition was explicitly seen as an essential element in the creation of a common market free from restraints on trade.}}
        \begin{itemize}
            \item[A.] Article 2 of the Rome Treaty provided inter alia that: 
            \begin{quote}
                “the Community shall promote, by creating a common market and the gradual harmonization of the economic policies of its Member States, a harmonious development of economic activities throughout the Community, a continuous and balanced expansion, increased stability, an increasingly rapid improvement in the standard of living and closer relations between States participating in it”
            \end{quote}
            \item[B.] The objective of protecting competition was explicitly stated in Article 3(f), according to which:
            \begin{quote}
                “For purposes of the previous article, the action of the Community shall include … the establishment of a system ensuring that competition is not distorted in the common market”. 
            \end{quote}
        \end{itemize}

        

    \subsection{Lisbon Treaty}

        1/12/2009 - With the entry into force of the Lisbon Treaty the framework described changed. 

        Thanks to the insistence of French President Sarkozy, the “end" of undistorted competition does not appear anymore to be one of the fundamental goals of the EU (at the time Europe was in a period of heavy crisis, and many politicians began to complain against the excesses of globalization, the free markets mantra, nihilist competition and backed the adoption of protectionist policies to “better serve” national interests, firms and employees). 
    
        Article 3 TEU (in the post-Lisbon formulation) now states that: 
        \begin{quote}
            “The Union shall establish an internal market. It shall work for the sustainable development of Europe based on balanced economic growth and price stability, a highly competitive social market economy, aiming at full employment and social progress, and a high level of protection and improvement of the quality of the environment. It shall promote scientific and technological advance”.
        \end{quote}

        \Note{
            \textcolor{royalpurple}{\textbf{The three-card monte by Mario Monti | }}
            Monti brought back through the window the principle that President Sarkozy attempted to throw out the door 
            He insisted that undistorted competition should at least be mentioned in a Protocol annexed to the TEU and to the TFEU. 
            Protocol 27 now points out that the internal market "includes a system which ensures that competition is not distorted". 
            However protocols must be regarded as an integral part of the Treaties themselves, i.e., protocols have the same value of the Treaties’ main articles. 
        }

        \Remark{
        The EEC Treaty contained, in particular, two provisions for the protection of competition in this field 
        \begin{itemize}
            \item Art. 85 EEC (then art. 81; now, after the Lisbon Treaty, we have Article 101 TFEU), which prohibits agreements, concerted decisions and concerted practices having the object or effect of restricting competition, save the possibility of exemption from the prohibition in the event of certain conditions; 
            \item Art. 86 (then 82, now Article 102 TFEU), which prohibits the abuse of a dominant position by one or more undertakings. 
        \end{itemize}
        }

\newpage
        \subsubsection{Article 101 TFUE, paragraph 1 (prohibition)}

            \begin{quote}
                1. The following shall be prohibited as incompatible with the internal market: all agreements between undertakings, decisions by associations of undertakings and concerted practices which may affect trade between Member States and which have as their object or effect the prevention, restriction or distortion of competition within the internal market, and in particular those which:
                    (a) directly or indirectly fix purchase or selling prices or any other trading conditions; (b) limit or control production, markets, technical development, or investment; (c) share markets or sources of supply; (d) apply dissimilar conditions to equivalent transactions with other trading parties, thereby placing them at a competitive disadvantage; (e) make the conclusion of contracts subject to acceptance by the other parties of supplementary obligations which, by their nature or according to commercial usage, have no connection with the subject of such contracts.
                2. Any agreements or decisions prohibited pursuant to this Article shall be automatically void.
                3. The provisions of paragraph 1 may, however, be declared inapplicable in the case of: - any agreement or category of agreements between undertakings, - any decision or category of decisions by associations of undertakings, - any concerted practice or category of concerted practices, which contributes to improving the production or distribution of goods or to promoting technical or economic progress, while allowing consumers a fair share of the resulting benefit, and which does not:
                    \begin{itemize}
                        \item[a.] impose on the undertakings concerned restrictions which are not indispensable to the attainment of these objectives;
                        \item[b.] afford such undertakings the possibility of eliminating competition in respect of a substantial part of the products in question.
                    \end{itemize}

            \end{quote}     

        \subsubsection{Article 102 (ex Article 82 TEC)}

            \begin{quote}
                Any abuse by one or more undertakings of a dominant position within the internal market or in a substantial part of it shall be prohibited as incompatible with the internal market in so far as it may affect trade between Member States.
                Such abuse may, in particular, consist in:
                    \begin{itemize}
                        \item[a.] directly or indirectly imposing unfair purchase or selling prices or other unfair trading conditions;
                        \item[b.] limiting production, markets or technical development to the prejudice of consumers;
                        \item[c.] applying dissimilar conditions to equivalent transactions with other trading parties, thereby placing them at a competitive disadvantage;
                        \item[d.] making the conclusion of contracts subject to acceptance by the other parties of supplementary obligations which, by their nature or according to commercial usage, have no connection with the subject of such contracts.
                    \end{itemize}
            \end{quote}

        \Note{
        \textbf{No Merger Control in the European Treaties; mergers are dealt with in a Regulation | }
        Neither the 1957 Treaty, nor the TEU and TFEU contain any rules on merger control. 
        The introduction of a merger control regime has been delayed until 1989 because of a series of conflicts and deep differences between Member States about the scope, objectives and procedures for assessing when a merger should be considered unlawful.
        The first merger regulation (Reg. n. 4064/89) has been modified several times and then repealed in 2004 by Reg. 139/2004. 
        Article 2, Reg. 139/2004 states that:
            \begin{quote}
                 “a concentration which would significantly impede effective competition, in the common market or in a substantial part of it, in particular as a result of the creation or strengthening of a dominant position, shall be declared incompatible with the common market”.
            \end{quote}
        }

        \Remark{
        The basic goals of articles 101 and 102, as currently interpreted, are broadly in line with the US ones. Competition law is intended to protect competitive pressure – or the process of competition – to maximize welfare, by prohibiting any conduct, by way of agreement or abuse of a dominant position by one or more firm, which may significantly reduce welfare. 
        }

\section{Internal Market integration and Competition Law}

    Some differences between US and EU are due to how the EU Treaties are written, and the end-goals (common market, employment, environment, etc.) which were and still are to be considered hierarchically superior to competition.

    In particular, EU competition policy has been influenced by the objective of integrating the various national realities into a single 'common' market (today, the internal market). 
    EU authorities have very harshly enforced the law against agreements capable of re-creating national barriers or hindering trade transactions between MS: 
    “an agreement between producer and distributor which might tend to restore the national divisions in trade between Member States might be such as to frustrate the Treaty’s objective of achieving the integration of national markets through the establishment of a single market. Thus (…) agreements aimed at partitioning national markets according to national borders or making the interpenetration of national markets more difficult, in particular those aimed at preventing or restricting parallel exports (should be considered) agreements whose object is to restrict competition within the meaning of that article of the Treaty” (ECJ, C-501/06 P, GlaxoSmithkline, § 61).


\newpage
    \subsection{Employment/Harmonious Development of the EU / Reduction of regional disparities}

        \Example{
        In the landmark decision Ford/Volkswagen, the EU Commission exempted a JV venture that eliminated competition between two world leading car manufacturers in the production of minivans’ platforms. 
        \textbf{Main procompetitive reason to exempt:} positive effects in terms of efficiency (construction of a very large-scale plant capable of reducing production cost). 
        \textbf{However}: the Commission also “takes note” of the fact that the project constitutes the largest ever single foreign investment in Portugal. It is estimated to lead, inter alia, to the creation of about 5,000 jobs and indirectly create up to another 10,000 jobs, as well as attracting other investment in the supply industry. It therefore contributes to the promotion of the harmonious development of the Community and the reduction of regional disparities which is one of the basic aims of the Treaty. It also furthers European market integration by linking Portugal more closely to the Community through one of its important industries (Comm., 23-12-1992, Ford / Volkswagen).
        }

    \subsection{Environmental vs Competition Policy}

        \Example{
        Philips/Osram formed a JV for the manufacture and sale of certain non-lead glass tubing (and components thereof) for incandescent and fluorescent lamps equipped with the necessary tools to reduce huge emission problems inherent in the manufacture of lead glass (at the time, lead, nitrogen oxide, antimony emissions were contributing to the so-called acid rain devastating German forests). 
        \textbf{The Commission exempted the JV because: }
        It will offer greater flexibility in quantities and types of product and a lower risk of breakdown and will have a production capacity substantially higher than that resulting from the combination of the production capacity of the facilities of the parent companies for the production of lead glass prior to the creation of the present JV. The JV will result in lower total energy usage and a better prospect of realizing energy reduction and waste emission programmes (Comm., 21-12-1994, Philips-Osram, § 25).
        }

    \subsection{Sustainability}

        The described scenario changed a couple of times. 
        \begin{itemize}
            \item The “More economic approach” shift, in early 2000, favoured an interpretation of the pros and cons of a firm’s conduct from a strict CS point of view (i.e., no other values to be considered such as pollution, employment, and the likes; short run focus).
            \item The new horizontal cooperation guidelines (July 2023) introduced a new way to appraise sustainability agreements, i.e. agreements that may foster a sustainable development. Sustainable development refers to the ability of society to consume and use the resources available today without compromising the ability of future generations to meet their own needs.
            \item It encompasses activities that support economic, environmental and social (including labour and human rights) development. The notion of sustainability objectives therefore includes addressing climate change, reducing pollution, limiting the use of natural resources, upholding human rights, ensuring a living income, fostering resilient infrastructure and innovation, reducing food waste, facilitating a shift to healthy and nutritious food, ensuring animal welfare, etc.
            \item 2024 and guidelines on exclusionary conduct. Focus on the competitive structure of the market. 
        \end{itemize}

    \subsection{State Aids}

        \Warning{
        We are not going to deal with State Aids legislation and case-law (lack of time, focus on firm’s behavior), nor on the FDI (competitiveness, market distorsions due to State intervention, etc.)
        }

        Articles 107-109 TFUE:
            \begin{quote}
                “any aid granted by a Member State or through State resources in any form whatsoever which distorts or threatens to distort competition by favouring certain undertakings or the production of certain goods shall, in so far as it affects trade between Member States, be incompatible with the internal market”
            \end{quote}